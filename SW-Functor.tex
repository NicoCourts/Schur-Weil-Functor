\documentclass[12pt]{article}

\usepackage{setspace}

\usepackage{amsmath, graphicx, color, fancyhdr, tikz-cd, mdframed, enumitem, framed, adjustbox, bbm, upgreek, xcolor, hyperref}
\usepackage[framed,thmmarks]{ntheorem}
\usepackage[style=alphabetic]{biblatex}
%Set the bibliography file
\bibliography{sources}

\usepackage[T1]{fontenc}
\usepackage[urw-garamond]{mathdesign}
\usepackage{garamondx}

%Replacement for the old geometry package
\usepackage{fullpage}

%Input my definitions
%set up theorem/definition/etc envs
%Problems will be created using their own counter and style
\theoremstyle{break}
\theoreminframepreskip{0pt}
\theoreminframepostskip{0pt}
\newframedtheorem{prob}{Problem}[section]

%solution template
\theoremstyle{nonumberbreak}
\theoremindent0.5cm
\theorembodyfont{\upshape}
\theoremseparator{:}
\theoremsymbol{\ensuremath\spadesuit}
\newtheorem{sol}{Solution}

%Theorems
\definecolor{thmcol}{RGB}{120,100,50}
\theoremstyle{changebreak}
\theoremseparator{}
\theoremsymbol{}
\theoremindent0.5cm
\theoremheaderfont{\color{thmcol}\bfseries} 
\newtheorem{thm}{Theorem}[subsection]

%Lemmas and Corollaries
\theoremheaderfont{\bfseries}
\newtheorem{lem}[thm]{Lemma}
\newtheorem{cor}[thm]{Corollary}
\newtheorem{prop}[thm]{Proposition}

%Create a new env that references a theorem and creates a 'primed' version
%Note this can be used recursively to get double, triple, etc primes
\newenvironment{thm-prime}[1]
  {\renewcommand{\thethm}{\ref{#1}$'$}%
   \addtocounter{thm}{-1}%
   \begin{thm}}
  {\end{thm}}

\setlength\fboxsep{15pt}

%Example
\theoremstyle{break}
\def\theoremframecommand{\colorbox[rgb]{0.9,0.9,0.9}}
\newshadedtheorem{ex}{Example}[section]

%Man, that's really good! Let's use the same thing for definitons.
\newenvironment{def-prime}[1]
  {\renewcommand{\thethm}{\ref{#1}$'$}%
   \addtocounter{thm}{-1}%
   \begin{def}}
  {\end{def}}

%proofs
\theoremstyle{nonumberbreak}
\theoremindent0.5cm
\theoremheaderfont{\sc}
\theoremseparator{}
\theoremsymbol{\ensuremath\spadesuit}
\newtheorem{prf}{Proof}

\theoremstyle{nonumberplain}
\theoremseparator{:}
\theoremsymbol{}
\newtheorem{conj}{Conjecture}

%remarks
\theoremstyle{change}
\theoremindent0.5cm
\theoremheaderfont{\sc}
\theoremseparator{:}
\theoremsymbol{}
\newtheorem{rmk}[thm]{Remark}

%Put page breaks before each part
\let\oldpart\part%
\renewcommand{\part}{\clearpage\oldpart}%

% Blackboard letters
\newcommand*{\bbA}{\mathbb{A}}
\newcommand*{\bbB}{\mathbb{B}}
\newcommand*{\bbC}{\mathbb{C}}
\newcommand*{\bbD}{\mathbb{D}}
\newcommand*{\bbE}{\mathbb{E}}
\newcommand*{\bbF}{\mathbb{F}}
\newcommand*{\bbG}{\mathbb{G}}
\newcommand*{\bbH}{\mathbb{H}}
\newcommand*{\bbI}{\mathbb{I}}
\newcommand*{\bbJ}{\mathbb{J}}
\newcommand*{\bbK}{\mathbb{K}}
\newcommand*{\bbL}{\mathbb{L}}
\newcommand*{\bbM}{\mathbb{M}}
\newcommand*{\bbN}{\mathbb{N}}
\newcommand*{\bbO}{\mathbb{O}}
\newcommand*{\bbP}{\mathbb{P}}
\newcommand*{\bbQ}{\mathbb{Q}}
\newcommand*{\bbR}{\mathbb{R}}
\newcommand*{\bbS}{\mathbb{S}}
\newcommand*{\bbT}{\mathbb{T}}
\newcommand*{\bbU}{\mathbb{U}}
\newcommand*{\bbV}{\mathbb{V}}
\newcommand*{\bbW}{\mathbb{W}}
\newcommand*{\bbX}{\mathbb{X}}
\newcommand*{\bbY}{\mathbb{Y}}
\newcommand*{\bbZ}{\mathbb{Z}}
%Fraktur letters
\newcommand*{\frakA}{\mathfrak{A}}
\newcommand*{\frakB}{\mathfrak{B}}
\newcommand*{\frakC}{\mathfrak{C}}
\newcommand*{\frakD}{\mathfrak{D}}
\newcommand*{\frakE}{\mathfrak{E}}
\newcommand*{\frakF}{\mathfrak{F}}
\newcommand*{\frakG}{\mathfrak{G}}
\newcommand*{\frakH}{\mathfrak{H}}
\newcommand*{\frakI}{\mathfrak{I}}
\newcommand*{\frakJ}{\mathfrak{J}}
\newcommand*{\frakK}{\mathfrak{K}}
\newcommand*{\frakL}{\mathfrak{L}}
\newcommand*{\frakM}{\mathfrak{M}}
\newcommand*{\frakN}{\mathfrak{N}}
\newcommand*{\frakO}{\mathfrak{O}}
\newcommand*{\frakP}{\mathfrak{P}}
\newcommand*{\frakQ}{\mathfrak{Q}}
\newcommand*{\frakR}{\mathfrak{R}}
\newcommand*{\frakS}{\mathfrak{S}}
\newcommand*{\frakT}{\mathfrak{T}}
\newcommand*{\frakU}{\mathfrak{U}}
\newcommand*{\frakV}{\mathfrak{V}}
\newcommand*{\frakW}{\mathfrak{W}}
\newcommand*{\frakX}{\mathfrak{X}}
\newcommand*{\frakY}{\mathfrak{Y}}
\newcommand*{\frakZ}{\mathfrak{Z}}
\newcommand*{\fraka}{\mathfrak{a}}
\newcommand*{\frakb}{\mathfrak{b}}
\newcommand*{\frakc}{\mathfrak{c}}
\newcommand*{\frakd}{\mathfrak{d}}
\newcommand*{\frake}{\mathfrak{e}}
\newcommand*{\frakf}{\mathfrak{f}}
\newcommand*{\frakg}{\mathfrak{g}}
\newcommand*{\frakh}{\mathfrak{h}}
\newcommand*{\fraki}{\mathfrak{i}}
\newcommand*{\frakj}{\mathfrak{j}}
\newcommand*{\frakk}{\mathfrak{k}}
\newcommand*{\frakl}{\mathfrak{l}}
\newcommand*{\frakm}{\mathfrak{m}}
\newcommand*{\frakn}{\mathfrak{n}}
\newcommand*{\frako}{\mathfrak{o}}
\newcommand*{\frakp}{\mathfrak{p}}
\newcommand*{\frakq}{\mathfrak{q}}
\newcommand*{\frakr}{\mathfrak{r}}
\newcommand*{\fraks}{\mathfrak{s}}
\newcommand*{\frakt}{\mathfrak{t}}
\newcommand*{\fraku}{\mathfrak{u}}
\newcommand*{\frakv}{\mathfrak{v}}
\newcommand*{\frakw}{\mathfrak{w}}
\newcommand*{\frakx}{\mathfrak{x}}
\newcommand*{\fraky}{\mathfrak{y}}
\newcommand*{\frakz}{\mathfrak{z}}
% Caligraphic letters
\newcommand*{\calA}{\mathcal{A}}
\newcommand*{\calB}{\mathcal{B}}
\newcommand*{\calC}{\mathcal{C}}
\newcommand*{\calD}{\mathcal{D}}
\newcommand*{\calE}{\mathcal{E}}
\newcommand*{\calF}{\mathcal{F}}
\newcommand*{\calG}{\mathcal{G}}
\newcommand*{\calH}{\mathcal{H}}
\newcommand*{\calI}{\mathcal{I}}
\newcommand*{\calJ}{\mathcal{J}}
\newcommand*{\calK}{\mathcal{K}}
\newcommand*{\calL}{\mathcal{L}}
\newcommand*{\calM}{\mathcal{M}}
\newcommand*{\calN}{\mathcal{N}}
\newcommand*{\calO}{\mathcal{O}}
\newcommand*{\calP}{\mathcal{P}}
\newcommand*{\calQ}{\mathcal{Q}}
\newcommand*{\calR}{\mathcal{R}}
\newcommand*{\calS}{\mathcal{S}}
\newcommand*{\calT}{\mathcal{T}}
\newcommand*{\calU}{\mathcal{U}}
\newcommand*{\calV}{\mathcal{V}}
\newcommand*{\calW}{\mathcal{W}}
\newcommand*{\calX}{\mathcal{X}}
\newcommand*{\calY}{\mathcal{Y}}
\newcommand*{\calZ}{\mathcal{Z}}
% Script Letters
\newcommand*{\scrA}{\mathscr{A}}
\newcommand*{\scrB}{\mathscr{B}}
\newcommand*{\scrC}{\mathscr{C}}
\newcommand*{\scrD}{\mathscr{D}}
\newcommand*{\scrE}{\mathscr{E}}
\newcommand*{\scrF}{\mathscr{F}}
\newcommand*{\scrG}{\mathscr{G}}
\newcommand*{\scrH}{\mathscr{H}}
\newcommand*{\scrI}{\mathscr{I}}
\newcommand*{\scrJ}{\mathscr{J}}
\newcommand*{\scrK}{\mathscr{K}}
\newcommand*{\scrL}{\mathscr{L}}
\newcommand*{\scrM}{\mathscr{M}}
\newcommand*{\scrN}{\mathscr{N}}
\newcommand*{\scrO}{\mathscr{O}}
\newcommand*{\scrP}{\mathscr{P}}
\newcommand*{\scrQ}{\mathscr{Q}}
\newcommand*{\scrR}{\mathscr{R}}
\newcommand*{\scrS}{\mathscr{S}}
\newcommand*{\scrT}{\mathscr{T}}
\newcommand*{\scrU}{\mathscr{U}}
\newcommand*{\scrV}{\mathscr{V}}
\newcommand*{\scrW}{\mathscr{W}}
\newcommand*{\scrX}{\mathscr{X}}
\newcommand*{\scrY}{\mathscr{Y}}
\newcommand*{\scrZ}{\mathscr{Z}}

%Section break
\newcommand*{\brk}{
\rule{2in}{.1pt}
}

%General purpose stuff
\DeclareMathOperator{\Aut}{Aut}
\DeclareMathOperator{\ch}{char}
\DeclareMathOperator{\rank}{rank}
\DeclareMathOperator{\End}{End}
\let\Im\relax
\DeclareMathOperator{\Im}{Im}

%Category Theory
\DeclareMathOperator{\Hom}{Hom}
\let\hom\relax
\DeclareMathOperator{\hom}{hom}
\DeclareMathOperator{\id}{id}
\DeclareMathOperator{\coker}{coker}
\DeclareMathOperator{\colim}{colim}
\DeclareMathOperator{\invlim}{\lim_{\leftarrow}}
\DeclareMathOperator{\dirlim}{\lim_{\rightarrow}}

%Commutative Algebra
\DeclareMathOperator{\gldim}{gldim}
\DeclareMathOperator{\projdim}{projdim}
\DeclareMathOperator{\injdim}{injdim}
\DeclareMathOperator{\findim}{findim}
\DeclareMathOperator{\flatdim}{flatdim}
\DeclareMathOperator{\depth}{depth}

%Common Categories
%\newcommand*{\modR}{\mathbf{mod}\text{-}R}
%\newcommand*{\Rmod}{R\text{-}\mathbf{mod}}
\newcommand{\rmod}[1]{\mathbf{mod}\text{-}#1}
\newcommand{\lmod}[1]{#1\text{-}\mathbf{mod}}
\DeclareMathOperator{\Vectk}{\mathbf{Vect}_k}
\DeclareMathOperator{\Ch}{\mathbf{Ch}}
\newcommand*{\Ab}{\mathbf{Ab}}
\newcommand*{\Grp}{\mathbf{Grp}}
\newcommand*{\Alg}{\mathbf{Alg}_k}
\newcommand*{\Ring}{\mathbf{Ring}}
\newcommand*{\K}{\mathbf{K}}
\newcommand*{\D}{\mathbf{D}}
\newcommand*{\Db}{\mathbf{D}^b}
\newcommand*{\Dpos}{\mathbf{D}^+}
\newcommand*{\Dneg}{\mathbf{D}^-}
\newcommand*{\Dbperf}{\mathbf{D}^b_{\text{perf}}}
\newcommand*{\Dsing}{\mathbf{D}_{sing}}
\newcommand{\CRing}{\mathbf{CRing}}
\DeclareMathOperator{\stmod}{\mathbf{stmod}}
\DeclareMathOperator{\StMod}{\mathbf{StMod}}
\DeclareMathOperator{\sHom}{\underline{Hom}}

%Homological algebra
\DeclareMathOperator{\cone}{cone}
\DeclareMathOperator{\HH}{HH}
\DeclareMathOperator{\Der}{Der}
\DeclareMathOperator{\Ext}{Ext}
\DeclareMathOperator{\Tor}{Tor}

%Lie algebras
\DeclareMathOperator{\ad}{ad}
\newcommand*{\gl}{\mathfrak{gl}}
\let\sl\relax
\newcommand*{\sl}{\mathfrak{sl}}
\let\sp\relax
\newcommand*{\sp}{\mathfrak{sp}}
\newcommand*{\so}{\mathfrak{so}}

% Hacks and Tweaks
% Enumerate will automatically use letters (e.g. part a,b,c,...)
\setenumerate[0]{label=(\alph*)}
% Always use wide tildes
\let\tilde\relax
\newcommand*{\tilde}[1]{\widetilde{#1}}
%raise that Chi!
\DeclareRobustCommand{\Chi}{{\mathpalette\irchi\relax}}
\newcommand{\irchi}[2]{\raisebox{\depth}{$#1\chi$}} 



%Shade definitions
\theoremindent0cm
\theoremheaderfont{\normalfont\bfseries} 
\def\theoremframecommand{\colorbox[rgb]{0.9,1,.8}}
\newshadedtheorem{defn}[thm]{Definition}

%%%%%%%%%%%%%%%%%%%%%%%%%%%%%%%%%%%%%%%%%%%%%%%%%%%%%%%%%%%%%%%%%%%%%%
%%%%%%%%%%%%%%%%%%%%%%% Customize Below %%%%%%%%%%%%%%%%%%%%%%%%%%%%%%
%%%%%%%%%%%%%%%%%%%%%%%%%%%%%%%%%%%%%%%%%%%%%%%%%%%%%%%%%%%%%%%%%%%%%%

%header stuff
\setlength{\headsep}{24pt}  % space between header and text
\pagestyle{fancy}     % set pagestyle for document
\lhead{Notes on the Schur-Weil Functor} % put text in header (left side)
\rhead{Nico Courts} % put text in header (right side)
\cfoot{\itshape p. \thepage}
\setlength{\headheight}{15pt}
%\allowdisplaybreaks

% Document-Specific Macros
\DeclareMathOperator{\1}{\mathbbm{1}}

\begin{document}
%make the title page
\title{Notes on the Grassmannian \vspace{-1ex}}
\author{Nico Courts}
\date{Summer 2019}
\maketitle

\begin{abstract}
	These notes are my summary of the development of the Schur-Weil functor from representations of the Schur algebra $S(n,d)$ to 
	representations of $\frakS_d$. After developing the theory that arose beginning in Schur's 1901 thesis, we will establish that this 
	functor is exact and behaves nicely with respect to simple modules, as well as being a monoidal functor (under the correct monoidal structure).
\end{abstract}

\section{Background and Introduction}
The primary reference for this paper is J.A. Green's \textit{Polynomial Representations of $GL_n$} \cite{green}. Other sources will be used as well, 
and will be introduced as needed.

The basic idea for this theory begins with Schur's thesis in 1901 \cite{schur-thesis}. Here he developed the theory underlying
the representation theory of the general linear group $GL_n$. As a tool, he recognized that the irreducible representations of 
$GL_n$ took on a particular form that made them amenable to admitting actions by symmetric groups. Then Schur could rely on 
Frobenius' development of the representation theory for $\frakS_d$ to say something about $GL_n$.

\subsection{Notation}
We always use $\frakS_d$ to denote the symmetric group on $d$ letters, and let $k$ be an infinite field (unless otherwise noted).

\subsection{Representations of Semigroups}
Recall that a \textbf{semigroup} is a group where we relax the inverse requirement. If we are interested in studying the representation 
theory of a semigroup $\Gamma$, there are some natural modules to consider:
\begin{ex}
	$k\Gamma=\bigoplus_{g\in \Gamma}kg$, the \textbf{semigroup algebra} for $\Gamma$ over $k$. Notice that if $\Gamma$ is infinite, we only take 
	the elements that are supported at finitely many $g$ (that is, only finitely many coefficients are nonzero).
\end{ex}
\begin{ex} $k^\Gamma$, the set of all maps $\Gamma\to k$, which is naturally a commutative $k$-algebra. It inherits a $k\Gamma$-bimodule structure 
	by considering the left- and right-translation maps $L_g:k^\Gamma\to k$ and $R_g:k^\Gamma\to k$ defined for all $g\in\Gamma$ by 
	\[L_g(f)(h)= f(gh)\qquad R_g(f)(h)=f(hg)\]
	Due to the fact that $L_g$ defines the \textit{right} action and $R_g$ the left, we write for simplicity:
	\[g\circ f=R_g(f)\quad\text{and}\quad f\circ g=L_g(f).\]
\end{ex}

It is in the context of the $k\Gamma$ module $k^\Gamma$ that we get our first definition:
\begin{defn}\label{defn-finitary}
	Let $k^\Gamma$ be as above. Then an element $f:\Gamma\to k$ is called a \textbf{finitary element} or \textbf{representative function}
	if $f$ satisfies one of the following equivalent conditions:
	\begin{itemize}
		\item $k^\Gamma f$ is finite dimensional over $k$.
		\item $f k^\Gamma$ is finite dimensional over $k$.
		\item $\Delta f\in k^\Gamma\otimes k^\Gamma$ -- in other words, the comultiplication $\Delta f$ is finitely supported.
	\end{itemize}
\end{defn}
\begin{rmk}
	The collection of all finitary elements of $k^\Gamma$ will be written $F=F(k^\Gamma)$ and is a $k$-bialgebra -- in particular,
	a sub-bialgebra of $k^\Gamma$.
\end{rmk}
\begin{rmk}
	The reason such elements are called representative functions is that if we fix a finite dimensional $V$ over $k$ 
	and a basis $\calB$ for $V$, we can write down the structure maps of a particular $\Gamma$ action by noticing 
	\[g\cdot \alpha=\sum_{\beta\in\calB}r_{\alpha\beta}(g)\beta\]
	where $r_{\alpha\beta}:\Gamma\to k$ are called the \textbf{coefficient functions} of the representation.

	Then the $k$-span $\operatorname{cf}(V)=\sum_{\alpha,\beta}k\cdot r_{\alpha\beta}$ of coefficient functions for a particular (finite-dimensional) representation 
	$V$ ends up being a subcoalgebra of $k^\Gamma$ and in fact every finitary function lies in $\operatorname{cf}(V)$ for some 
	finite-dimensional $V$. Here we can just take $V=k^\Gamma f$ which is finite dimensional by defn~\ref{defn-finitary}.
\end{rmk}

An important idea that comes from this that one can define an \textbf{algebraic representation theory} of $\Gamma$ over $k$. Take any subcoalgebra $A$ of $F=F(k^\Gamma)$
and define the $A$\textbf{-representation theory} of $\Gamma$ to be the full subcategory $\mathbf{mod}_A(k\Gamma)$ of $\rmod (k\Gamma)$ to be the one whose 
objects are the (finite-dimensional) $k\Gamma$ modules $V$ such that $\text{cf}(V)\subseteq A$.

A nice result is that the (left) $A$-rational $k\Gamma$-modules are equivalent to the category $\mathbf{com}(A)$ of finite (right) $A$-comodules.
\begin{prop}
	$\mathbf{mod}_A(k\Gamma)\simeq \mathbf{com}(A)$
\end{prop}
\begin{prf}
	\textit{Sketch:} Take an $A$-rational $k\Gamma$ module $V$ with action $\tau:\Gamma\to V$. Fix a basis $\{v_i\}_1^n$
	and as before extract the structure maps $r_{\alpha\beta}$:
	\[\tau(v_\alpha)=\sum_\beta r_{\alpha\beta} v_\beta.\]
	
	Then define the comodule $(V,\gamma)$ with coaction $\gamma$ via 
	\[\gamma(v_\alpha)=\sum_\beta v_\beta\otimes r_{\alpha\beta}.\]
	One can check this defines a comodule structure on $V$. 

	The process reverses very easily, where one extracts the $r_{\alpha\beta}$ from the coaction.
	By construction we will have that $r_{\alpha\beta}\in A$, so we get the $A$-rationality for free.

	Some mopping up shows that these are equivalences of categories, which is pretty believable. 
\end{prf}	
\begin{rmk}
	One of the most immediate consequences of this realization (it is barely a proof) is that we can define a left $A^\ast=\Hom_k(A,k)$-module (note that $A^\ast$ is a $k$-algebra with the convolution product)
	structure on any right $A$-comodule. We do this by writing for any $f\in A^\ast$
	\[f\cdot v=(\1_V\otimes f)(\gamma(v))\]
	or in coordinates 
	\[f\cdot v_\alpha=\sum_\beta f(r_{\alpha\beta})v_\beta\]
\end{rmk}

\section{\texorpdfstring{$GL_n$}{GLn} and the Schur Algebra}
From now on, we specialize the theory in the section above to the case when $\Gamma=GL_n$ (we can think of this as the group scheme). 
Then we can set $A_k$ to be the $k$-algebra generated by the function $c_{ij}$ which extracts the $(i,j)^{th}$ entry.
This gives us precisely the polynomial representations of $\Gamma$ and restricting to the homogenous polynomials in $A$ of degree $d$
gives us $A(n,d)$.

\begin{defn}
	The \textbf{Schur algebra} $S(n,d)$ is the dual of $A=A(n,d):$
	\[S(n,d)=A^\ast=\Hom_k(A(n,d),k).\]
\end{defn}

%%%%%%%%%%%%%%%%%%%%%%%%%%%%%%%%%%%%%%
%%%%%%%%%%  Bibliography %%%%%%%%%%%%%
%%%%%%%%%%%%%%%%%%%%%%%%%%%%%%%%%%%%%%
\medskip

\printbibliography

\end{document}