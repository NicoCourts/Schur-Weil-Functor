\documentclass[12pt]{article}

\usepackage{setspace}

\usepackage{amsmath, graphicx, color, fancyhdr, tikz-cd, mdframed, enumitem, framed, adjustbox, bbm, upgreek, xcolor, hyperref}
\usepackage[framed,thmmarks]{ntheorem}
\usepackage[style=alphabetic, bibencoding=utf8]{biblatex}
%Set the bibliography file
\bibliography{sources}

\usepackage[T1]{fontenc}
\usepackage[urw-garamond]{mathdesign}
\usepackage{garamondx}

%Replacement for the old geometry package
\usepackage{fullpage}

%Input my definitions
%set up theorem/definition/etc envs
%Problems will be created using their own counter and style
\theoremstyle{break}
\theoreminframepreskip{0pt}
\theoreminframepostskip{0pt}
\newframedtheorem{prob}{Problem}[section]

%solution template
\theoremstyle{nonumberbreak}
\theoremindent0.5cm
\theorembodyfont{\upshape}
\theoremseparator{:}
\theoremsymbol{\ensuremath\spadesuit}
\newtheorem{sol}{Solution}

%Theorems
\definecolor{thmcol}{RGB}{120,100,50}
\theoremstyle{changebreak}
\theoremseparator{}
\theoremsymbol{}
\theoremindent0.5cm
\theoremheaderfont{\color{thmcol}\bfseries} 
\newtheorem{thm}{Theorem}[subsection]

%Lemmas and Corollaries
\theoremheaderfont{\bfseries}
\newtheorem{lem}[thm]{Lemma}
\newtheorem{cor}[thm]{Corollary}
\newtheorem{prop}[thm]{Proposition}

%Create a new env that references a theorem and creates a 'primed' version
%Note this can be used recursively to get double, triple, etc primes
\newenvironment{thm-prime}[1]
  {\renewcommand{\thethm}{\ref{#1}$'$}%
   \addtocounter{thm}{-1}%
   \begin{thm}}
  {\end{thm}}

\setlength\fboxsep{15pt}

%Example
\theoremstyle{break}
\def\theoremframecommand{\colorbox[rgb]{0.9,0.9,0.9}}
\newshadedtheorem{ex}{Example}[section]

%Man, that's really good! Let's use the same thing for definitons.
\newenvironment{def-prime}[1]
  {\renewcommand{\thethm}{\ref{#1}$'$}%
   \addtocounter{thm}{-1}%
   \begin{def}}
  {\end{def}}

%proofs
\theoremstyle{nonumberbreak}
\theoremindent0.5cm
\theoremheaderfont{\sc}
\theoremseparator{}
\theoremsymbol{\ensuremath\spadesuit}
\newtheorem{prf}{Proof}

\theoremstyle{nonumberplain}
\theoremseparator{:}
\theoremsymbol{}
\newtheorem{conj}{Conjecture}

%remarks
\theoremstyle{change}
\theoremindent0.5cm
\theoremheaderfont{\sc}
\theoremseparator{:}
\theoremsymbol{}
\newtheorem{rmk}[thm]{Remark}

%Put page breaks before each part
\let\oldpart\part%
\renewcommand{\part}{\clearpage\oldpart}%

% Blackboard letters
\newcommand*{\bbA}{\mathbb{A}}
\newcommand*{\bbB}{\mathbb{B}}
\newcommand*{\bbC}{\mathbb{C}}
\newcommand*{\bbD}{\mathbb{D}}
\newcommand*{\bbE}{\mathbb{E}}
\newcommand*{\bbF}{\mathbb{F}}
\newcommand*{\bbG}{\mathbb{G}}
\newcommand*{\bbH}{\mathbb{H}}
\newcommand*{\bbI}{\mathbb{I}}
\newcommand*{\bbJ}{\mathbb{J}}
\newcommand*{\bbK}{\mathbb{K}}
\newcommand*{\bbL}{\mathbb{L}}
\newcommand*{\bbM}{\mathbb{M}}
\newcommand*{\bbN}{\mathbb{N}}
\newcommand*{\bbO}{\mathbb{O}}
\newcommand*{\bbP}{\mathbb{P}}
\newcommand*{\bbQ}{\mathbb{Q}}
\newcommand*{\bbR}{\mathbb{R}}
\newcommand*{\bbS}{\mathbb{S}}
\newcommand*{\bbT}{\mathbb{T}}
\newcommand*{\bbU}{\mathbb{U}}
\newcommand*{\bbV}{\mathbb{V}}
\newcommand*{\bbW}{\mathbb{W}}
\newcommand*{\bbX}{\mathbb{X}}
\newcommand*{\bbY}{\mathbb{Y}}
\newcommand*{\bbZ}{\mathbb{Z}}
%Fraktur letters
\newcommand*{\frakA}{\mathfrak{A}}
\newcommand*{\frakB}{\mathfrak{B}}
\newcommand*{\frakC}{\mathfrak{C}}
\newcommand*{\frakD}{\mathfrak{D}}
\newcommand*{\frakE}{\mathfrak{E}}
\newcommand*{\frakF}{\mathfrak{F}}
\newcommand*{\frakG}{\mathfrak{G}}
\newcommand*{\frakH}{\mathfrak{H}}
\newcommand*{\frakI}{\mathfrak{I}}
\newcommand*{\frakJ}{\mathfrak{J}}
\newcommand*{\frakK}{\mathfrak{K}}
\newcommand*{\frakL}{\mathfrak{L}}
\newcommand*{\frakM}{\mathfrak{M}}
\newcommand*{\frakN}{\mathfrak{N}}
\newcommand*{\frakO}{\mathfrak{O}}
\newcommand*{\frakP}{\mathfrak{P}}
\newcommand*{\frakQ}{\mathfrak{Q}}
\newcommand*{\frakR}{\mathfrak{R}}
\newcommand*{\frakS}{\mathfrak{S}}
\newcommand*{\frakT}{\mathfrak{T}}
\newcommand*{\frakU}{\mathfrak{U}}
\newcommand*{\frakV}{\mathfrak{V}}
\newcommand*{\frakW}{\mathfrak{W}}
\newcommand*{\frakX}{\mathfrak{X}}
\newcommand*{\frakY}{\mathfrak{Y}}
\newcommand*{\frakZ}{\mathfrak{Z}}
\newcommand*{\fraka}{\mathfrak{a}}
\newcommand*{\frakb}{\mathfrak{b}}
\newcommand*{\frakc}{\mathfrak{c}}
\newcommand*{\frakd}{\mathfrak{d}}
\newcommand*{\frake}{\mathfrak{e}}
\newcommand*{\frakf}{\mathfrak{f}}
\newcommand*{\frakg}{\mathfrak{g}}
\newcommand*{\frakh}{\mathfrak{h}}
\newcommand*{\fraki}{\mathfrak{i}}
\newcommand*{\frakj}{\mathfrak{j}}
\newcommand*{\frakk}{\mathfrak{k}}
\newcommand*{\frakl}{\mathfrak{l}}
\newcommand*{\frakm}{\mathfrak{m}}
\newcommand*{\frakn}{\mathfrak{n}}
\newcommand*{\frako}{\mathfrak{o}}
\newcommand*{\frakp}{\mathfrak{p}}
\newcommand*{\frakq}{\mathfrak{q}}
\newcommand*{\frakr}{\mathfrak{r}}
\newcommand*{\fraks}{\mathfrak{s}}
\newcommand*{\frakt}{\mathfrak{t}}
\newcommand*{\fraku}{\mathfrak{u}}
\newcommand*{\frakv}{\mathfrak{v}}
\newcommand*{\frakw}{\mathfrak{w}}
\newcommand*{\frakx}{\mathfrak{x}}
\newcommand*{\fraky}{\mathfrak{y}}
\newcommand*{\frakz}{\mathfrak{z}}
% Caligraphic letters
\newcommand*{\calA}{\mathcal{A}}
\newcommand*{\calB}{\mathcal{B}}
\newcommand*{\calC}{\mathcal{C}}
\newcommand*{\calD}{\mathcal{D}}
\newcommand*{\calE}{\mathcal{E}}
\newcommand*{\calF}{\mathcal{F}}
\newcommand*{\calG}{\mathcal{G}}
\newcommand*{\calH}{\mathcal{H}}
\newcommand*{\calI}{\mathcal{I}}
\newcommand*{\calJ}{\mathcal{J}}
\newcommand*{\calK}{\mathcal{K}}
\newcommand*{\calL}{\mathcal{L}}
\newcommand*{\calM}{\mathcal{M}}
\newcommand*{\calN}{\mathcal{N}}
\newcommand*{\calO}{\mathcal{O}}
\newcommand*{\calP}{\mathcal{P}}
\newcommand*{\calQ}{\mathcal{Q}}
\newcommand*{\calR}{\mathcal{R}}
\newcommand*{\calS}{\mathcal{S}}
\newcommand*{\calT}{\mathcal{T}}
\newcommand*{\calU}{\mathcal{U}}
\newcommand*{\calV}{\mathcal{V}}
\newcommand*{\calW}{\mathcal{W}}
\newcommand*{\calX}{\mathcal{X}}
\newcommand*{\calY}{\mathcal{Y}}
\newcommand*{\calZ}{\mathcal{Z}}
% Script Letters
\newcommand*{\scrA}{\mathscr{A}}
\newcommand*{\scrB}{\mathscr{B}}
\newcommand*{\scrC}{\mathscr{C}}
\newcommand*{\scrD}{\mathscr{D}}
\newcommand*{\scrE}{\mathscr{E}}
\newcommand*{\scrF}{\mathscr{F}}
\newcommand*{\scrG}{\mathscr{G}}
\newcommand*{\scrH}{\mathscr{H}}
\newcommand*{\scrI}{\mathscr{I}}
\newcommand*{\scrJ}{\mathscr{J}}
\newcommand*{\scrK}{\mathscr{K}}
\newcommand*{\scrL}{\mathscr{L}}
\newcommand*{\scrM}{\mathscr{M}}
\newcommand*{\scrN}{\mathscr{N}}
\newcommand*{\scrO}{\mathscr{O}}
\newcommand*{\scrP}{\mathscr{P}}
\newcommand*{\scrQ}{\mathscr{Q}}
\newcommand*{\scrR}{\mathscr{R}}
\newcommand*{\scrS}{\mathscr{S}}
\newcommand*{\scrT}{\mathscr{T}}
\newcommand*{\scrU}{\mathscr{U}}
\newcommand*{\scrV}{\mathscr{V}}
\newcommand*{\scrW}{\mathscr{W}}
\newcommand*{\scrX}{\mathscr{X}}
\newcommand*{\scrY}{\mathscr{Y}}
\newcommand*{\scrZ}{\mathscr{Z}}

%Section break
\newcommand*{\brk}{
\rule{2in}{.1pt}
}

%General purpose stuff
\DeclareMathOperator{\Aut}{Aut}
\DeclareMathOperator{\ch}{char}
\DeclareMathOperator{\rank}{rank}
\DeclareMathOperator{\End}{End}
\let\Im\relax
\DeclareMathOperator{\Im}{Im}

%Category Theory
\DeclareMathOperator{\Hom}{Hom}
\let\hom\relax
\DeclareMathOperator{\hom}{hom}
\DeclareMathOperator{\id}{id}
\DeclareMathOperator{\coker}{coker}
\DeclareMathOperator{\colim}{colim}
\DeclareMathOperator{\invlim}{\lim_{\leftarrow}}
\DeclareMathOperator{\dirlim}{\lim_{\rightarrow}}

%Commutative Algebra
\DeclareMathOperator{\gldim}{gldim}
\DeclareMathOperator{\projdim}{projdim}
\DeclareMathOperator{\injdim}{injdim}
\DeclareMathOperator{\findim}{findim}
\DeclareMathOperator{\flatdim}{flatdim}
\DeclareMathOperator{\depth}{depth}

%Common Categories
%\newcommand*{\modR}{\mathbf{mod}\text{-}R}
%\newcommand*{\Rmod}{R\text{-}\mathbf{mod}}
\newcommand{\rmod}[1]{\mathbf{mod}\text{-}#1}
\newcommand{\lmod}[1]{#1\text{-}\mathbf{mod}}
\DeclareMathOperator{\Vectk}{\mathbf{Vect}_k}
\DeclareMathOperator{\Ch}{\mathbf{Ch}}
\newcommand*{\Ab}{\mathbf{Ab}}
\newcommand*{\Grp}{\mathbf{Grp}}
\newcommand*{\Alg}{\mathbf{Alg}_k}
\newcommand*{\Ring}{\mathbf{Ring}}
\newcommand*{\K}{\mathbf{K}}
\newcommand*{\D}{\mathbf{D}}
\newcommand*{\Db}{\mathbf{D}^b}
\newcommand*{\Dpos}{\mathbf{D}^+}
\newcommand*{\Dneg}{\mathbf{D}^-}
\newcommand*{\Dbperf}{\mathbf{D}^b_{\text{perf}}}
\newcommand*{\Dsing}{\mathbf{D}_{sing}}
\newcommand{\CRing}{\mathbf{CRing}}
\DeclareMathOperator{\stmod}{\mathbf{stmod}}
\DeclareMathOperator{\StMod}{\mathbf{StMod}}
\DeclareMathOperator{\sHom}{\underline{Hom}}

%Homological algebra
\DeclareMathOperator{\cone}{cone}
\DeclareMathOperator{\HH}{HH}
\DeclareMathOperator{\Der}{Der}
\DeclareMathOperator{\Ext}{Ext}
\DeclareMathOperator{\Tor}{Tor}

%Lie algebras
\DeclareMathOperator{\ad}{ad}
\newcommand*{\gl}{\mathfrak{gl}}
\let\sl\relax
\newcommand*{\sl}{\mathfrak{sl}}
\let\sp\relax
\newcommand*{\sp}{\mathfrak{sp}}
\newcommand*{\so}{\mathfrak{so}}

% Hacks and Tweaks
% Enumerate will automatically use letters (e.g. part a,b,c,...)
\setenumerate[0]{label=(\alph*)}
% Always use wide tildes
\let\tilde\relax
\newcommand*{\tilde}[1]{\widetilde{#1}}
%raise that Chi!
\DeclareRobustCommand{\Chi}{{\mathpalette\irchi\relax}}
\newcommand{\irchi}[2]{\raisebox{\depth}{$#1\chi$}} 



%Shade definitions
\theoremindent0cm
\theoremheaderfont{\normalfont\bfseries} 
\def\theoremframecommand{\colorbox[rgb]{0.9,1,.8}}
\newshadedtheorem{defn}[thm]{Definition}

%%%%%%%%%%%%%%%%%%%%%%%%%%%%%%%%%%%%%%%%%%%%%%%%%%%%%%%%%%%%%%%%%%%%%%
%%%%%%%%%%%%%%%%%%%%%%% Customize Below %%%%%%%%%%%%%%%%%%%%%%%%%%%%%%
%%%%%%%%%%%%%%%%%%%%%%%%%%%%%%%%%%%%%%%%%%%%%%%%%%%%%%%%%%%%%%%%%%%%%%

%header stuff
\setlength{\headsep}{24pt}  % space between header and text
\pagestyle{fancy}     % set pagestyle for document
\lhead{Notes on the Schur-Weyl Functor} % put text in header (left side)
\rhead{Nico Courts} % put text in header (right side)
\cfoot{\itshape p. \thepage}
\setlength{\headheight}{15pt}
%\allowdisplaybreaks

% Document-Specific Macros
\DeclareMathOperator{\1}{\mathbbm{1}}

\begin{document}
%make the title page
\title{Notes on the Schur-Weyl Functor \vspace{-1ex}}
\author{Nico Courts}
\date{Summer 2019}
\maketitle

\begin{abstract}
	These notes are my summary of the development of the Schur-Weyl functor from representations of the Schur algebra $S(n,d)$ to 
	representations of $\frakS_d$. After developing the theory that arose beginning in Schur's 1901 thesis, we will establish that this 
	functor is exact and behaves nicely with respect to simple modules, as well as being a monoidal functor (under the correct monoidal structure).
\end{abstract}

\section{Background and Introduction}
The primary reference for this paper is J.A. Green's \textit{Polynomial Representations of $GL_n$} \cite{green}. Other sources will be used as well, 
and will be introduced as needed.

The basic idea for this theory begins with Schur's thesis in 1901 \cite{schur-thesis}. Here he developed the theory underlying
the representation theory of the general linear group $GL_n$. As a tool, he recognized that the irreducible representations of 
$GL_n$ took on a particular form that made them amenable to admitting actions by symmetric groups. Then Schur could rely on 
Frobenius' development of the representation theory for $\frakS_d$ to say something about $GL_n$.

\subsection{Notation}
We always use $\frakS_d$ to denote the symmetric group on $d$ letters, and let $k$ be an infinite field (unless otherwise noted).

\subsection{Representations of Semigroups}
Recall that a \textbf{semigroup} is a group where we relax the inverse requirement. If we are interested in studying the representation 
theory of a semigroup $\Gamma$, there are some natural modules to consider:
\begin{ex}
	$k\Gamma=\bigoplus_{g\in \Gamma}kg$, the \textbf{semigroup algebra} for $\Gamma$ over $k$. Notice that if $\Gamma$ is infinite, we only take 
	the elements that are supported at finitely many $g$ (that is, only finitely many coefficients are nonzero).
\end{ex}
\begin{ex} $k^\Gamma$, the set of all maps $\Gamma\to k$, which is naturally a commutative $k$-algebra. It inherits a $k\Gamma$-bimodule structure 
	by considering the left- and right-translation maps $L_g:k^\Gamma\to k$ and $R_g:k^\Gamma\to k$ defined for all $g\in\Gamma$ by 
	\[L_g(f)(h)= f(gh)\qquad R_g(f)(h)=f(hg)\]
	Due to the fact that $L_g$ defines the \textit{right} action and $R_g$ the left, we write for simplicity:
	\[g\circ f=R_g(f)\quad\text{and}\quad f\circ g=L_g(f).\]
\end{ex}

It is in the context of the $k\Gamma$ module $k^\Gamma$ that we get our first definition:
\begin{defn}\label{defn-finitary}
	Let $k^\Gamma$ be as above. Then an element $f:\Gamma\to k$ is called a \textbf{finitary element} or \textbf{representative function}
	if $f$ satisfies one of the following equivalent conditions:
	\begin{itemize}
		\item $k^\Gamma f$ is finite dimensional over $k$.
		\item $f k^\Gamma$ is finite dimensional over $k$.
		\item $\Delta f\in k^\Gamma\otimes k^\Gamma$ -- in other words, the comultiplication $\Delta f$ is finitely supported.
	\end{itemize}
\end{defn}
\begin{rmk}
	The collection of all finitary elements of $k^\Gamma$ will be written $F=F(k^\Gamma)$ and is a $k$-bialgebra -- in particular,
	a sub-bialgebra of $k^\Gamma$.
\end{rmk}
\begin{rmk}
	The reason such elements are called representative functions is that if we fix a finite dimensional $V$ over $k$ 
	and a basis $\calB$ for $V$, we can write down the structure maps of a particular $\Gamma$ action by noticing 
	\[g\cdot \alpha=\sum_{\beta\in\calB}r_{\alpha\beta}(g)\beta\]
	where $r_{\alpha\beta}:\Gamma\to k$ are called the \textbf{coefficient functions} of the representation.

	Then the $k$-span $\operatorname{cf}(V)=\sum_{\alpha,\beta}k\cdot r_{\alpha\beta}$ of coefficient functions for a particular (finite-dimensional) representation 
	$V$ ends up being a subcoalgebra of $k^\Gamma$ and in fact every finitary function lies in $\operatorname{cf}(V)$ for some 
	finite-dimensional $V$. Here we can just take $V=k^\Gamma f$ which is finite dimensional by defn~\ref{defn-finitary}.
\end{rmk}

An important idea that comes from this that one can define an \textbf{algebraic representation theory} of $\Gamma$ over $k$. Take any subcoalgebra $A$ of $F=F(k^\Gamma)$
and define the $A$\textbf{-representation theory} of $\Gamma$ to be the full subcategory $\mathbf{mod}_A(k\Gamma)$ of $\rmod (k\Gamma)$ to be the one whose 
objects are the (finite-dimensional) $k\Gamma$ modules $V$ such that $\text{cf}(V)\subseteq A$.

A nice result is that the (left) $A$-rational $k\Gamma$-modules are equivalent to the category $\mathbf{com}(A)$ of finite (right) $A$-comodules.
\begin{prop}\label{prop-modcomod}
	$\mathbf{mod}_A(k\Gamma)\simeq \mathbf{com}(A)$
\end{prop}
\begin{prf}
	\textit{Sketch:} Take an $A$-rational $k\Gamma$ module $V$ with action $\tau:\Gamma\to V$. Fix a basis $\{v_i\}_1^n$
	and as before extract the structure maps $r_{\alpha\beta}$:
	\[\tau(v_\alpha)=\sum_\beta r_{\alpha\beta} v_\beta.\]
	
	Then define the comodule $(V,\gamma)$ with coaction $\gamma$ via 
	\[\gamma(v_\alpha)=\sum_\beta v_\beta\otimes r_{\alpha\beta}.\]
	One can check this defines a comodule structure on $V$. 

	The process reverses very easily, where one extracts the $r_{\alpha\beta}$ from the coaction.
	By construction we will have that $r_{\alpha\beta}\in A$, so we get the $A$-rationality for free.

	Some mopping up shows that these are equivalences of categories, which is pretty believable. 
\end{prf}	
\begin{rmk}
	One of the most immediate consequences of this realization (it is barely a proof) is that we can define a left $A^\ast=\Hom_k(A,k)$-module (note that $A^\ast$ is a $k$-algebra with the convolution product)
	structure on any right $A$-comodule. We do this by writing for any $f\in A^\ast$
	\[f\cdot v=(\1_V\otimes f)(\gamma(v))\]
	or in coordinates 
	\[f\cdot v_\alpha=\sum_\beta f(r_{\alpha\beta})v_\beta\]
\end{rmk}

\section{Polynomial representations of \texorpdfstring{$GL_n$}{GLn} and the Schur Algebra}
From now on, we specialize the theory in the section above to the case when $\Gamma=GL_n$ (we can think of this as the group scheme). 

\subsection{Generators and Bases}
There are very natural choices of generators (as (co)algebras and as $k$ spaces). Starting off, we can define
\[A_k(n)\cong k[\omega_{ij}]\]
for $1\le i,j\le n$, where we can think of $\omega_{ij}:\Gamma\to k$ as the map that ``extracts'' from $g\in\Gamma$ the $(i,j)^{th}$ entry.

Then $A_k(n)$ is the set of \textbf{polynomial maps} $\Gamma\to k$. Then we define $A_k(n,r)$ to be the $k$-space spanned by all homogeneous degree $r$ 
polynomials in $A_k(n)$. A stars-and-bars argument gets us that there are $\binom{n^2+r-1}{r}$ ways to choose a monomial of degree $r$ from among the $n^2$ generators,
so this gives us the dimension of $A_k(n,d)$.

\subsection{Index Notation}
We need to use multi-indicies to rigorously discuss monomials so let $I(n,r)$ denote the set of length $r$ multi-indices drawn from $\underline n$. Then 
we can impose the equivalence relation: if $\alpha,\beta\in I(n,r)$,
\[\alpha\sim\beta \quad\Leftrightarrow\quad \alpha=(n_{i_1},\dots,n_{i_r})=(n_{i_{\sigma(1)}},\dots, n_{i_{\sigma(r)}})=\beta\]
for some $\sigma\in\frakS_r$. That is, two multi-indices are equivalent if they contain the same indicies with the same multiplicities. Sometimes it is useful to think 
of this as a group action of $\frakS_r$ on $I(n,r)$, and we say two elements are equivalent if they are in the same orbit.

Extend this group action (and thus relation) to $I(n,r)\times I(n,r)$ where $G$ acts diagonally:
\[g\cdot(i,j)=(g\cdot i,g\cdot j).\]

\subsection{Module Categories}
The module categories\footnote{Equivalently the appropriate comodule categories in light of prop.~\ref{prop-modcomod}} $\mathbf{mod}_{A}(k\Gamma)$ of 
(homogeneous degree $r$) polynomial representations of $GL_n(k)$ (when $A=A_k(n,r)$ and $A=A_k(n)$ respectively) are precisely what you'd think. We will sometimes 
use the notation $M_k(n)$ and $M_k(n,r)$ following the text.

One of the big realizations of Schur's was that any polynomial representation of $GL_n$ splits into a direct sum of homogeneous polynomial representations:
\begin{thm}[{\cite[p.5]{schur-thesis}}]
	Let $V\in M_k(n)$. Then 
	\[V\cong \bigoplus_{r\ge 0} V_r\]
	where $V_r\in M_k(n,r)$ for all $r$.
\end{thm}
\begin{cor}
	The indecomposable modules in $M_k(n)$ are all in $M_k(n,r)$ for some $r$.
\end{cor}

But then if we are interested in the structure of polynomial representations of $\Gamma$, we can restrict our attention entirely to the homogeneous modules
of any particular degree. The upshot here is that $A_k(n,r)$ is finite dimensional, so very computationally amenable.

\subsection{The Schur Algebra}
\begin{defn}
	The \textbf{Schur algebra} $S_k(n,d)$ is the dual of $A=A_k(n,d):$
	\[S_k(n,d)=A^\ast=\Hom_k(A(n,d),k).\]
\end{defn}

We can write down an explicit basis for $S_k(n,r)$ as a dual basis for the basis $\{c_{ij}|i,j\in I(n,r)\}$ for $A_k(n,r)$. We denote this basis $\xi_{ij}$, where 
\[\xi_{ij}(c_{kl})=\begin{cases}
	1,& (i,j)\sim (k,l)\\
	0,&\text{otherwise}
\end{cases}\]
and we can define multiplication via the coproduct $\Delta$ on $A_k(s,r)$:
\[\xi\eta(c)=\sum \xi(c_{(1)})\eta(c_{(2)})\]
where we recall that $\Delta(c_{ij})=\sum_k c_{ik}\otimes c_{kj}$ and $\varepsilon(c)=c(I_n)$ are the coalgebra structure inherited on $A_k(n,r)$ 
from $A_k(n)$ and thus from $k^\Gamma$.

\begin{prop}
	The product of two basis elements in $S_k(n,r)$ is given by 
	\[\xi_{ab}\xi_{cd}=\sum_{(p,q)\in I(n,r)^2}Z_{a,b,c,d,p,q}\xi_{pq}\]
	where 
	\[Z_{a,b,c,d,p,q}=\#\{s\in I(n,r)|(a,b)\sim(p,s)\text{ and }(c,d)\sim(s,q)\}\]
\end{prop}
\begin{cor}
	$\xi_{ab}\xi_{cd}=0$ if $b\not\sim c$.
\end{cor}
\begin{cor}
	$\xi_{ab}\xi_{bb}=\xi_{ab}=\xi_{aa}\xi_{ab}$.
\end{cor}

The last thing here is to notice that $\varepsilon=\sum_{I(n,r)^2}\xi_{ab}$ where equality can be seen by evaluating at the basis $c_{ab}$ of $A_k(n,r)$.

\subsection{The Evaluation Map}
There is a natural map $e:k\Gamma\to S_k(n,r)$ defined in the following way: if $f\in A_k(n,r)\subset k^\Gamma$ (which can be uniquely extended $k$-linearly 
to a map on $k\Gamma$),
\[e(\kappa)(f)=f(\kappa)\]
where $\kappa\in k\Gamma$.

The use of this map is that (since it can be shown to be surjective and that things in the kernel of this map must act by zero on $M_k(n,r)$), that the category
of $S_k(n,r)$ modules and the category of $A_k(n,r)$-rational $k\Gamma$ representations are equivalent. The evaluation map gives us a direct translation between 
the two actions: $\kappa\cdot v=e(\kappa)\cdot v$ and if $k\Gamma$ acts on $V$ with structure maps $(r_{ab})$, then for $\xi\in S_k(n,r)$,
\[\xi\cdot v_b=\sum_a \xi(r_{ab})v_a\]

\subsection{Modular Theory}
We can extract the $\bbZ$ forms $A_\bbZ(n)$ and $A_\bbZ(n,r)$ as the $\bbZ$ span of the basis elements $c_{ij}^\bbQ$ (the basis with respect to $k=\bbQ$). These 
are closed under $\Delta$ and have the further property that for instance $\varepsilon(A_\bbZ(n,r))\subseteq\bbZ$.

Then we can extend scalars in the way we'd like: there is a $k$-colagebra isomorphism 
\[A_\bbZ(n,r)\otimes_\bbZ k\cong A_k(n,r),\] 
so we can recover the larger algebra from this $\bbZ$-form. But this actually transfers to the Schur algebra as well!

Define $S_\bbZ(n,r)$ to be the collection of $\xi\in S_\bbQ(n,r)$ such that $\xi(A_\bbZ(n,r))\subseteq\bbZ$. Then we get a $k$-algebra isomorphism
\[S_\bbZ(n,r)\otimes_\bbZ k\cong S_k(n,r).\]
This gives us a $\bbZ$-form for the scheme $S(n,r)$ -- or that ``the Schur scheme is defined over $\bbZ$.''

We can also define a $\bbZ$-form of a ($A_\bbQ(n,r)$-rational) $\bbQ\Gamma$-module $V$ to be denoted $V_\bbZ$ and to be the $\bbZ$-span of a $\bbQ$-basis for $V$
and to furthermore be closed under the $S_\bbZ(n,r)$-action (leveraging the equivalence of categories in the previous section).

This process of generating $V_k$ modules by extending scalars from a $\bbZ$-form is called \textbf{modular reduction}, and has a small caveat:
the $\bbZ$-form we pick needn't be unique and, in general, using different $\bbZ$-forms will yield non-isomorphic extensions to $k$-modules. The upshot here,
following the theory developed in \cite{brauer} and \cite{green-locFinReps}, is that the multiplicity with which simple modules occur in a composition series of $V_k$
are not dependent on the choice of $\bbZ$-form.

\subsection{The Trick}
To do some magic we notice that there is a duality at play between the $A_k(n,r)$-regular representations of $GL_n$ and the representations of $\frakS_r$, 
and to see this we consider the natural action of these groups on $E^{\otimes r}$, where $E\cong k^n$. The left action by $\Gamma$ is just the diagonal action 
on each $k^n$ and $\frakS_r$ acts on $E^{\otimes r}$ on the right by permuting the tensor factors. The critical thing to notice here is that these two 
actions commute with one another:
\[(g\cdot v)\cdot\sigma=g\cdot(v\cdot\sigma),\ \forall g\in\Gamma, \sigma\in\frakS_r.\]

Then we get some results from Schur's 1927 paper \textit{\"Uber the rationalen Darstellungen der allgemeinen linearen Gruppe}, a reference to which I am yet to cook up.
\begin{thm}[Schur]
	Let $\varphi:S_k(n,r)\to \End_k(E^{\otimes r})$ be the representation afforded by the above action of $S_k(n,r)$. Then 
	\begin{itemize}
		\item $\Im \varphi=\End_{k\frakS_r}(E^{\otimes r})$; and 
		\item $\ker\varphi=0$.
	\end{itemize}
	So $S_k(n,r)\cong \End_{k\frakS_r}(E^{\otimes r})$.
\end{thm}
The idea here is to take our $\xi_{ab}$ basis for $S_k(n,r)$ and determine where the images go. Not too hard, I think.
\begin{cor}[Schur]
	If $\ch k=0$ or $\ch k>r$, then $S_k(n,r)$ is semisimple. That is every element of $M_k(n,r)$ is completely reducible.
\end{cor}
To see this, we just use the fact that $k\frakS_r$ is semisimple as in usual representation theory. Thus $E^{\otimes r}$ is 
completely reducible, but the endomorphism ring of a completely reducible module is semisimple, so by the above theorem, $S_k(n,r)$ is semisimple.

\subsection{Back to \texorpdfstring{$\bbZ$}{Z}-forms}
In fact, $E^{\otimes r}$ is a $GL_n$-module (this time regarded as an affine group scheme over $\bbZ$). Let $\{V_k\}$ be a collection 
of $k$-modules for each $k$ in some family of infinite fields. Then we say $\{V_k\}$ is \textbf{defined over $\bbZ$} if there is a $\bbZ$-form $V_\bbZ$
and a family of isomorphisms $\delta_k:V_\bbZ\otimes k\to V_k$.

In particular, the family of modules $E^{\otimes r}_k$ is defined over $\bbZ$.

\subsection{Contravariant Duality}
In group theory a usual way to define a dual representation for a $k\Gamma$-module $V$ is to take the linear 
dual $V^\ast=\Hom_k(V,k)$ and define a left action of $\Gamma$ by 
\[g\cdot f(v)=f(g^{-1}v)\]
which give you another (left) $k\Gamma$-module. The problem with this process in our case is that the resulting 
module may have structure maps $r_{ab}$ that no longer lie in $A_k(n,r)$. To account for this, we instead define the 
\textbf{contravariant dual} $V^\circ$, defined using the transpose:
\[g\cdot f=f(g^{\text{tr}}v).\]

It is clear that the map $J:S_k(n,r)\to S_k(n,r)$ sending $\xi_{ab}$ to $\xi_{ba}$ is an involutory antihomomorphism.
Notice 
\[J(\xi)(c_{ab})=\xi(c_{ba})\]
and so by taking $\xi=e_g$, we get 
\[J(e_g)(c_{ab})=e_g(c_{ba})=c_{ba}(g)=c_{ab}(g^{\text{tr}})\]
so $J(e_g)=e_{g^{\text{tr}}}$.

So given an $A_k(n,r)$-rational $k\Gamma$ representation $V$, the corresponding $S_k(n,r)$ action on $V^\circ$ is given by 
\[\xi\cdot f (v)=f(J(\xi)v).\]


%%%%%%%%%%%%%%%%%%%%%%%%%%%%%%%%%%%%%%
%%%%%%%%%%  Bibliography %%%%%%%%%%%%%
%%%%%%%%%%%%%%%%%%%%%%%%%%%%%%%%%%%%%%
\medskip

\printbibliography

\end{document}